\documentclass[twoside, a4paper]{article}
\usepackage{enumitem}


\usepackage[utf8]{inputenc}
\usepackage[T1]{fontenc}
\usepackage{textcomp}
\usepackage{gensymb}
\renewcommand{\vec}[1]{\mathbf{#1}}

\usepackage[style=alphabetic,sorting=ynt]{biblatex} % (mvl: run with biber, a utf8 aware biblatex implentation)
\addbibresource{bibliography.bib}

\title{Model Checking Petri Nets using BDDs}
\author{Leeuwen, M. van\\
  \texttt{s0180858}\\
  \and
  Voortman, N.E.F.\\
  \texttt{s1247247}
}

\begin{document}
\maketitle

\section{Introduction}
This report discusses the implementation of a CTL model checker for Petri Nets using BDDs. The implementation uses the multi-core decision diagram library Sylvan \parencite{sylvan}. A set of Petri Net models is used to validate the correctness of the implementation. These models that are used are from the Model Checking Contest \parencite{mcc:2017}. For a small sample of models the implementation gives results for satcount and CTL model checking that are consistent with the results from reference implementations.

\subsection{Implemented Features}

The features that are implemented are a follows:
\begin{enumerate}
\item Reachability
\item SatCount
\item CTL Model Checking
\end{enumerate}
\section{Representing a petri net as BDD}
Input to the model checker are ANDL files. An ANDL file specifies the places and transitions in a petri net. The model checker parses ANDL files and maintains a representation that can be used to later create description diagrams. The parser maintains a hashmap of transitions and places in an \texttt{andl\_context\_t} type, defined in \texttt{andl.h}. The places hashmaps uses \texttt{places\_struct\_t} which maps the place name to a BDD variable and an initial marking. The transitions hashmap maps transition names to \texttt{transitions\_struct\_t}, which contains arrays of incoming arc and outgoing arcs. An arc is then an integer which is the BDD variable of the place the arc relates to.

The structure in \texttt{andl\_context\_t} is later used to construct BDDs that represent the inital marking and the transitions of the petri net. These BDDs are required to find all reachable states of the petri net (which in itself is expressed as a new bdd \texttt{v}), and to count the number of minterms for this bdd. The file \texttt{petri-helpers.h} contains a number of functions help create these BDDs using an \texttt{andl\_context\_t}. A BDD representing the fireability of a petri net transition isconstructed using \texttt{petri\_fireable\_transition}. When applied to all known transitions, this results in an array of all BDD transition relations. Before this step it is required to sort in- and out-arc arrays of transitions using \texttt{petri\_sort\_arcs}. For a number of places $N$, \texttt{petri\_fireable\_transition} constructs a BDD of even BDD variables $0..N$, whenever $n$ is an in-arc it is required that the non-prime variable is marked and the prime variable is marked. Whenever $n$ is an out-arc, it is required that the non-prime variable is unmarked and the prime variable is marked. Whenever $n$ is not an arc, it is required that the non-prime variable and prime variable are equal.



\section{How to navigate the code}
\begin{description}[align=left]
\item [bdd-util.c:50] explain
\item [petri-util.c:50] explain
\end{description}


\section{Results}
Here we present the acquired of the implementation on a number of test cases.

\begin{tabular}[pos]{l | r | r}
  Model & SATCount & CTL check \\
  \hline
  Philosophers & 243 & a\\
\end{tabular}


\printbibliography
\end{document}
